\documentclass{article}

\usepackage{amsmath,amssymb}

\usepackage{microtype}
\usepackage[utf8]{inputenc} 
\usepackage[T1]{fontenc}
\usepackage{fourier}

\usepackage{fullpage}

\DeclareMathOperator{\Spec}{Spec}
\DeclareMathOperator{\rk}{rk}

\newcommand{\ZZ}{\mathbb{Z}}
\newcommand{\FF}{\mathbb{F}}
\newcommand{\QQ}{\mathbb{Q}}
\newcommand{\RR}{\mathbb{R}}
\newcommand{\CC}{\mathbb{C}}
\renewcommand{\AA}{\mathbb{A}}
\renewcommand{\Re}{\operatorname{Re}}

\usepackage[numbers]{natbib}

\usepackage{xcolor}

\usepackage[hidelinks]{hyperref}

\hypersetup{
    colorlinks,
    linkcolor={red!60!black},
    citecolor={blue!60!black},
    urlcolor={blue!80!black}
}

\author{Alexey Beshenov}
\date{March 2021}

\title{Research project on Weil-étale cohomology}

\begin{document}

\maketitle

In this note I will briefly describe what is Weil-étale cohomology,
my contributions to its study, and plans for further research.

\section{Zeta functions of schemes}

Given a scheme $X$ of finite type over $\Spec \ZZ$, one may attach to it the
corresponding \textbf{zeta function}
$$\zeta_X (s) = \prod_{x \in |X|} \frac{1}{1 - N (x)^{-s}}.$$
Here $|X|$ denotes the set of closed points of $X$, and for $x \in |X|$ the norm
$N (x)$ is the order of the corresponding residue field
$\kappa (x) = \mathcal{O}_{X,x}/\mathfrak{m}_{X,x}$. The above product converges
for $\Re s > \dim X$, and conjecturally, the zeta function admits a meromorphic
continuation to the whole complex plane.

In particular, if $X = \Spec \mathcal{O}_F$ is the spectrum of the ring of
integers of a number field $F/\QQ$, then $\zeta_X (s) = \zeta_F (s)$ is the
Dedekind zeta function studied extensively in algebraic number theory. If $X$ is
a smooth projective variety over a finite field $\FF_q$, then
$\zeta_X (s) = Z_X (q^{-s})$, where
$$Z_X (t) = \exp \Bigl(\sum_{n\ge 1} \frac{\# X (\FF_{q^n})}{n}\,t^n\Bigr)$$
is the Hasse--Weil zeta function, whose basic properties are given by Weil
conjectures in algebraic geometry (see e.g. \cite{Katz-Motives}).

For basic facts and conjectures about zeta functions of schemes, I refer to
Serre's survey \cite{Serre-65}.

Of particular interest are the so-called \textbf{special values} of
$\zeta_X (s)$ at integers $s = n \in \ZZ$. Namely, if $d_n$ is the vanishing
order of $\zeta_X (s)$ at $s = n$, then the corresponding special value is
defined to be the leading Taylor coefficient at $s = n$:
$$\zeta_X^* (n) = \lim_{s \to n} (s - n)^{d_n}\,\zeta_X (s)$$
(assuming the meromorphic continuation around $s = n$).

\section{Weil-étale cohomology}

Both vanishing order $d_n$ and special value $\zeta_X^* (n)$ are expected to
have expressions in terms of \emph{certain invariants} attached to $X$. There
are various conjectures, of varying generality, that make this precise.  I am
interested in a relatively recent \textbf{Weil-étale cohomology} program,
initiated by Stephen Lichtenbaum
\cite{Lichtenbaum-05,Lichtenbaum-09-Euler,Lichtenbaum-09-number-rings}.  Other
results for the case of varieties over finite fields have been obtained by
Geisser \cite{Geisser-2004,Geisser-2006}.

Let me briefly explain what one expects from Weil-étale cohomology. Let $X$ be a
separated scheme of finite type over $\Spec \ZZ$. Then for a fixed integer $n$,
Weil-étale cohomology consists of abelian groups $H^i_{W,c} (X,\ZZ(n))$ with the
following conjectural properties.

\begin{enumerate}
\item[W1)] $H^i_{W,c} (X,\ZZ(n))$ are finitely generated abelian groups, trivial
  for $|i| \gg 0$.

  As a consequence, to these groups one can associate the corresponding
  \textbf{determinant} $\det_\ZZ H^\bullet_{W,c} (X, \ZZ(n))$ in the sense of
  \cite{Knudsen-Mumford-1976}, which is a free $\ZZ$-module of rank $1$.

\item[W2)] After tensoring these cohomology groups with $\RR$, one obtains a long
  exact sequence of finite-dimensional real vector spaces
  $$\cdots \to H^{i-1}_{W,c} (X,\ZZ(n)) \otimes \RR \xrightarrow{\smile\theta} H^i_{W,c} (X,\ZZ(n)) \otimes \RR \xrightarrow{\smile\theta} H^{i+1}_{W,c} (X,\ZZ(n)) \otimes \RR \to \cdots$$

  By well-known properties of determinants of complexes, this induces a
  \emph{canonical} isomorphism
  $$\lambda\colon \RR \xrightarrow{\cong} \Bigl(\det_\ZZ H^\bullet_{W,c} (X, \ZZ(n))\Bigr) \otimes \RR.$$

\item[W3)] The vanishing order of $\zeta_X (s)$ at $s = n \in \ZZ$ is
  conjecturally given by
  $$d_n = \sum_{i\in \ZZ} (-1)^i \cdot i \cdot \rk_\ZZ H^i_{W,c} (X,\ZZ(n)).$$

\item[W4)] The corresponding special value is determined up to sign by
  $$\lambda (\zeta_X^* (n)^{-1})\cdot \ZZ = \det_\ZZ H^\bullet_{W,c} (X, \ZZ (n)).$$
\end{enumerate}

Baptiste Morin gave in \cite{Morin-2014} a construction of Weil-étale cohomology
for $X$ a separated scheme of finite type, proper and regular, and $n =
0$. Later on this construction was generalized together with Matthias Flach in
\cite{Flach-Morin-2018} to any $n \in \ZZ$, under the same assumptions on $X$
(for $n > 0$ the formula W4) is corrected by a rational factor $C(X,n)$ defined
in \cite[\S 5.4]{Flach-Morin-2018}).

\section{My work on Weil-étale cohomology}

In my PhD thesis \cite{These}, co-supervised by Baptiste Morin and Bas
Edixhoven, I generalized the work of Flach and Morin to any $X$ that is
separated and of finite type over $\Spec \ZZ$ (thus removing the assumption that
$X$ is proper or regular), while considering the case of $n < 0$
(which turns out to simplify certain aspects of the theory).

The main building block of $H^i_{W,c} (X,\ZZ(n))$ is the (étale)
\textbf{motivic cohomology} $H^i_\text{\it ét} (X, \ZZ^c (n))$, defined in terms
of ``dualizing cycle complexes'' $\ZZ^c (n)$, as introduced by Geisser in
\cite{Geisser-2010}, and an \textbf{arithmetic duality theorem} in terms of
$\ZZ^c (n)$, also due to Geisser (ibid.).

The precise constructions are quite technical, so I refer to my preprints
\cite{Weil-etale-preprint-1,Weil-etale-preprint-2} for further details.
What is important is that
\begin{itemize}
\item the property W1) above is established assuming finite generation of
  motivic cohomology $H^i_\text{\it ét} (X, \ZZ^c (n))$;

\item the sequence of real vector spaces in W2) is defined in terms of the
  \textbf{regulator} map; the exactness in W2) follows from the standard
  conjecture about the regulator.
\end{itemize}

% \section{Why Weil-étale cohomology?}

As most of the formulas for special values, all this is hugely conjectural,
especially at the level of generality we are interested in. A compelling
evidence in favor of the above special value conjecture is that, whenever the
comparison makes sense, it is equivalent to the \textbf{Tamagawa number
  conjecture} (\textbf{TNC}) of Bloch--Kato--Fontaine--Perrin-Riou
\cite{Fontaine-Perrin-Riou}.

One new interesting point is the following. If $Z \subset X$ is a closed
subscheme and $U = X\setminus Z$ is its open complement, then
$\zeta_X (s) = \zeta_Z (s) \, \zeta_U (s)$. Accordingly, one should expect the
special value conjecture to be compatible with such ``open-closed
decompositions'' of schemes: the special value conjecture for $X$ should be
equivalent to the corresponding conjecture for $Z$ and $U$. I prove in my
thesis that this is indeed the case in my situation: morally, this comes from a
long exact sequence
\[ \cdots \to H^i_{W,c} (U,\ZZ(n)) \to H^i_{W,c} (X,\ZZ(n)) \to H^i_{W,c} (Z,\ZZ(n)) \to H^{i+1}_{W,c} (U,\ZZ(n)) \to \cdots \]
Similarly, it is not difficult to see that one has
$\zeta_{\AA^r_X} (s) = \zeta_X (s-r)$, and thus a special value conjecture for
the affine bundle $\AA^r_X$ at $s = n$ should be equivalent to the corresponding
conjecture for $X$ at $s = n-r$. I~prove that this is also the case.

As a result, one can start from certain very special cases of $X$ for which the
special value conjecture is known (e.g. using the equivalence with TNC and known
cases of the latter), and then build new schemes (not necessarily proper or
regular) for which the conjecture holds as well, unconditionally. I refer to
\cite{Weil-etale-preprint-2} for precise unconditional results.

\section{My research projects}

Here I will list various problems that I have in mind for my future work.

\begin{enumerate}
\item It is interesting to write down specific formulas for particular cases of
  $X$ that follow from the Weil-étale formalism.

  For instance, Jordan and Poonen write down in \cite{Jordan-Poonen-2020} a
  formula for $\zeta_X^* (1)$, where $X$ is any affine one-dimensional scheme of
  finite type over $\Spec \ZZ$. Using my machinery, similar formulas can be
  produced for $\zeta_X^* (n)$, with $X$ a one-dimensional separated scheme of
  finite type, and $n < 0$.

  This is the project I am currently finishing (see a working draft
  \href{https://cadadr.org/papers/1-dim-schemes.pdf}{cadadr.org/papers/1-dim-schemes.pdf} for further details).
  Its continuation concens a similar formula for $n > 0$, which corresponds to a
  very different situation.

\item It is likely that the results of \cite{Weil-etale-preprint-1} and
  \cite{Weil-etale-preprint-2} can be generalized to Weil-étale cohomology for
  $\ZZ$-constructible sheaves $\mathcal{F}$, in the spirit of
  \cite{Geisser-Suzuki-2020}. I plan to consider these generalizations in the
  near future.

\item The actual construction of Weil-étale cohomology is not formulated in
  terms of separate groups $H^i_{W,c} (X, \ZZ (n))$, but in terms of complexes
  $R\Gamma_{W,c} (X, \ZZ (n))$. At present, these are defined (both in
  \cite{Flach-Morin-2018} and \cite{Weil-etale-preprint-1}) up to a
  \emph{non-unique} isomorphism in the derived category $\mathbf{D} (\ZZ)$,
  as a mapping fiber of certain canonical morphism in $\mathbf{D} (\ZZ)$.
  This is not a big issue for the special value conjecture, since the
  determinants
  $\det_\ZZ R\Gamma_{W,c} (X, \ZZ (n)) = \det_\ZZ H^\bullet_{W,c} (X, \ZZ(n))$
  are uniquely defined, but nevertheless, it would be useful to find a more
  canonical definition for $R\Gamma_{W,c} (X, \ZZ (n))$.

  This could be probably remedied using dg-categories \cite{Toen-2011} or
  Lurie's stable $\infty$-categories \cite{Lurie-stable-oo}.

\item In the above exposition, I swept under the rug some details about
  regulators. I use the construction of Kerr, Lewis, and Müller-Stach from
  \cite{KLMS}. In general, the regulator has to do with complex points
  $X (\CC)$, and the regulator in \cite{KLMS} is defined for $X_\CC$ being
  smooth and quasi-projective. This is quite unfortunate, since my construction
  of Weil-étale cohomology $H^i_{W,c} (X, \ZZ (n))$ works for any $X$ that is
  separated and of finite type over $\Spec \ZZ$ (assuming finite generation of
  the corresponding motivic cohomology). It would be interesting to find
  appropriate generalizations of the regulator for singular $X_\CC$, and connect
  these to my machinery.

  There are variants of motivic cohomology for singular complex varieties,
  defined in terms of hyperresolutions, e.g. Hanamura's
  ``Chow cohomology groups'' \cite{Hanamura-2000}, that allow to formally extend
  the formula of Kerr--Lewis--Müller-Stach to the singular case. However,
  it seems like the singular regulators haven't been considered thoroughly for
  special value conjectures.

\item The Tamagawa number conjecture has a generalization, known as the
  \textbf{equivariant Tamagawa number conjecture} (\textbf{ETNC});
  see e.g. Flach's survey \cite{Flach-ETNC}. Similarly, Weil-étale cohomology
  should also have an ``equivariant refinement'', and it would be interesting to
  write it down for my construction, and prove its compatibility with ETNC.
\end{enumerate}

These are a few specific questions that naturally arise from my previous
work. In general, Weil-étale cohomology is an active research topic, and there
are still many open problems related to it.

\pagebreak

\bibliographystyle{amsalpha-cust}
\bibliography{zeta}

\end{document}
