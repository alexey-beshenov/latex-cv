\documentclass{article}

\usepackage{amsmath,amssymb}

\usepackage{microtype}
\usepackage[utf8]{inputenc} 
\usepackage[T1]{fontenc}
\usepackage{fourier}

\usepackage{fullpage}

\DeclareMathOperator{\Spec}{Spec}
\DeclareMathOperator{\rk}{rk}

\newcommand{\ZZ}{\mathbb{Z}}
\newcommand{\FF}{\mathbb{F}}
\newcommand{\QQ}{\mathbb{Q}}
\newcommand{\RR}{\mathbb{R}}
\newcommand{\CC}{\mathbb{C}}
\renewcommand{\AA}{\mathbb{A}}
\renewcommand{\Re}{\operatorname{Re}}

\usepackage[numbers]{natbib}

\usepackage{xcolor}

\usepackage[hidelinks]{hyperref}

\hypersetup{
    colorlinks,
    linkcolor={red!60!black},
    citecolor={blue!60!black},
    urlcolor={blue!80!black}
}

\author{Alexey Beshenov}
\date{November 2021}

\title{Research project on Weil-étale cohomology}

\begin{document}

\maketitle

In this note I will briefly describe what is Weil-étale cohomology,
my contributions to its study, and plans for further research.

\section{Zeta functions of schemes}

Given a scheme $X$ of finite type over $\Spec \ZZ$, one may attach to it the
corresponding \textbf{zeta function}
$$\zeta_X (s) = \prod_{x \in |X|} \frac{1}{1 - N (x)^{-s}}.$$
Here $|X|$ denotes the set of closed points of $X$, and for $x \in |X|$ the norm
$N (x)$ is the order of the corresponding residue field
$\kappa (x) = \mathcal{O}_{X,x}/\mathfrak{m}_{X,x}$. The above product converges
for $\Re s > \dim X$, and it is conjectured that the zeta function admits a
meromorphic continuation to the whole complex plane.

In particular, if $X = \Spec \mathcal{O}_F$ is the spectrum of the ring of
integers of a number field $F/\QQ$, then $\zeta_X (s) = \zeta_F (s)$ is the
Dedekind zeta function, which has been extensively studied in algebraic number
theory. If $X$ is a smooth projective variety over a finite field $\FF_q$, then
$\zeta_X (s) = Z_X (q^{-s})$, where
$$Z_X (t) = \exp \Bigl(\sum_{n\ge 1} \frac{\# X (\FF_{q^n})}{n}\,t^n\Bigr)$$
is the Hasse--Weil zeta function, whose basic properties are given by Weil
conjectures in algebraic geometry (see e.g. \cite{Katz-Motives}).

For basic facts and conjectures about zeta functions of schemes I refer to
Serre's survey \cite{Serre-65}.

Of particular interest are the so-called \textbf{special values} of
$\zeta_X (s)$ at integers $s = n \in \ZZ$. Namely, if $d_n$ is the vanishing
order of $\zeta_X (s)$ at $s = n$, then the corresponding special value is
defined as the leading Taylor coefficient at $s = n$:
$$\zeta_X^* (n) = \lim_{s \to n} (s - n)^{d_n}\,\zeta_X (s)$$
(assuming the meromorphic continuation around $s = n$).

\section{Weil-étale cohomology}

It is expected that both the vanishing order $d_n$ and the special value
$\zeta_X^* (n)$ can be expressed by \emph{certain invariants} of $X$. There are
many conjectures of varying generality that make this precise. I am
interested in a relatively recent \textbf{Weil-étale cohomology} program,
initiated by Stephen Lichtenbaum
\cite{Lichtenbaum-05,Lichtenbaum-09-Euler,Lichtenbaum-09-number-rings}. Other
results for the case of varieties over finite fields were obtained by
Geisser \cite{Geisser-2004,Geisser-2006}.

Let me briefly explain what one expects from Weil-étale cohomology. Let $X$ be a
separated scheme of finite type over $\Spec \ZZ$. Then for a fixed integer $n$,
the Weil-étale cohomology consists of abelian groups $H^i_{W,c} (X,\ZZ(n))$ with
the following conjectural properties.

\begin{enumerate}
\item[W1)] $H^i_{W,c} (X,\ZZ(n))$ are finitely generated abelian groups, trivial
  for $|i| \gg 0$.
 
  Consequently, one can assign to these groups the corresponding
  \textbf{determinant} $\det_\ZZ H^\bullet_{W,c} (X, \ZZ(n))$ in the sense of
  Knudsen and Mumford \cite{Knudsen-Mumford-1976}, which is a free $\ZZ$-module
  of rank $1$.

\item[W2)] After tensoring these cohomology groups with $\RR$, one obtains a long
  exact sequence of finite-dimensional real vector spaces
  $$\cdots \to H^{i-1}_{W,c} (X,\ZZ(n)) \otimes \RR \xrightarrow{\smile\theta} H^i_{W,c} (X,\ZZ(n)) \otimes \RR \xrightarrow{\smile\theta} H^{i+1}_{W,c} (X,\ZZ(n)) \otimes \RR \to \cdots$$

  By well-known properties of determinants of complexes, this induces a
  \emph{canonical} isomorphism
  $$\lambda\colon \RR \xrightarrow{\cong} \Bigl(\det_\ZZ H^\bullet_{W,c} (X, \ZZ(n))\Bigr) \otimes \RR.$$

\item[W3)] The vanishing order of $\zeta_X (s)$ at $s = n \in \ZZ$ is
  conjecturally given by
  $$d_n = \sum_{i\in \ZZ} (-1)^i \cdot i \cdot \rk_\ZZ H^i_{W,c} (X,\ZZ(n)).$$

\item[W4)] The corresponding special value is determined up to sign by
  $$\lambda (\zeta_X^* (n)^{-1})\cdot \ZZ = \det_\ZZ H^\bullet_{W,c} (X, \ZZ (n)).$$
\end{enumerate}

Baptiste Morin gave in \cite{Morin-2014} a construction of Weil-étale cohomology
for $X$ a separated scheme of finite type, proper and regular, and $n =
0$. Later this construction was generalized together with Matthias Flach in
\cite{Flach-Morin-2018} to any $n \in \ZZ$, under the same assumptions on $X$
(for $n > 0$ the formula W4) is corrected by a rational factor $C(X,n)$, defined
in [ibid. \S 5.4]).

\section{My work on Weil-étale cohomology}

In my doctoral thesis \cite{These}, which was co-supervised by Baptiste Morin
and Bas Edixhoven, I generalized the work of Flach and Morin to any $X$ that is
separated and of finite type over $\Spec \ZZ$ (thus removing the assumption that
$X$ is proper or regular) for the case of $n < 0$ (which, as it turns out,
simplifies certain aspects of the theory).

The main building block of $H^i_{W,c} (X,\ZZ(n))$ is the (étale)
\textbf{motivic cohomology} $H^i_\text{\it ét} (X, \ZZ^c (n))$, defined in terms
of ``dualizing cycle complexes'' $\ZZ^c (n)$, as considered by Geisser in
\cite{Geisser-2010} in the context of \textbf{arithmetic duality theorems}.

The exact constructions are quite technical, so I refer the reader to my
preprints \cite{Weil-etale-cohomology,Weil-etale-zeta-values} for more
details. It is important to note that
\begin{itemize}
\item the above property W1) is established assuming the finite generation of
  étale motivic cohomology $H^i_\text{\it ét} (X, \ZZ^c (n))$;

\item the sequence of real vector spaces in W2) is defined via a
  \textbf{regulator} map; the exactness in W2) follows from Beilinson
  conjectures about regulators.
\end{itemize}

% \section{Why Weil-étale cohomology?}

Like most of the formulas for special values, this is all conjectural,
especially at the level of generality we are interested in. A compelling
evidence in favor of the above special value conjecture is that, whenever the
comparison makes sense, it is equivalent to the \textbf{Tamagawa number
  conjecture} (\textbf{TNC}) of Bloch--Kato--Fontaine--Perrin-Riou
\cite{Fontaine-Perrin-Riou}.

A new interesting point is the following. If $Z \subset X$ is a closed subscheme
and $U = X\setminus Z$ is its open complement, then
$\zeta_X (s) = \zeta_Z (s) \, \zeta_U (s)$. Accordingly, one should expect the
special value conjecture to be compatible with such ``closed-open
decompositions'' of schemes: the special value conjecture for $X$ should be
equivalent to the corresponding conjecture for $Z$ and $U$. I prove in
\cite{Weil-etale-zeta-values} that this is indeed the case in my situation:
morally, this follows from a long exact sequence
\[ \cdots \to H^i_{W,c} (U,\ZZ(n)) \to H^i_{W,c} (X,\ZZ(n)) \to H^i_{W,c} (Z,\ZZ(n)) \to H^{i+1}_{W,c} (U,\ZZ(n)) \to \cdots \]
Similarly, it is not hard to see that one has
$\zeta_{\AA^r_X} (s) = \zeta_X (s-r)$, and hence a special value conjecture for
the affine bundle $\AA^r_X$ at $s = n$ should be equivalent to the corresponding
conjecture for $X$ at $s = n-r$. I~prove in \cite{Weil-etale-zeta-values} that
this is also the case.

As a result, we can start from certain very special cases of $X$ for which the
special value conjecture is known (e.g. using equivalence with TNC and known
cases of TNC), and then construct new schemes (not necessarily proper or
regular) for which the conjecture also holds, unconditionally. I refer the
reader to \cite{Weil-etale-zeta-values} for precise unconditional results.

In a recent preprint \cite{Weil-etale-1-dim}, I was able to compute the
Weil-étale cohomology of an arbitrary one-dimensional separated scheme of finite
type $X \to \Spec \ZZ$ and $n < 0$, and as a consequence, obtain a new formula
for the special values $\zeta_X^* (n)$ for $n < 0$ in terms the étale motivic
cohomology of $X$ and a regulator. This was motivated in part by the work of
Jordan and Poonen, who worked out a formula for $\zeta_X^* (1)$ for $X$
one-dimensional, reduced and affine in \cite{Jordan-Poonen-2020}.  My formula
from \cite{Weil-etale-1-dim} holds unconditionally if for all generic points
$\eta \in X$ with $\operatorname{char} \kappa (\eta) = 0$ the corresponding
extension $\kappa (\eta) / \QQ$ is abelian.

\section{My research plans}

Here I will list several problems that I have in mind for my future work.

\begin{enumerate}
\item It is an interesting project to extend the results of
  \cite{Weil-etale-1-dim} to all integers $n \in \ZZ$.

  In the work of Flach and Morin \cite{Flach-Morin-2018} the corresponding
  special value conjecture for $n > 0$ involves a ``correction factor''
  $C (X,n) \in \QQ$. Subsequent work
  \cite{Flach-Morin-2020,Flach-Morin-2020-Muenster} sheds some light on its
  nature. The key to generalizing Flach and Morin's work for non-regular or
  non-proper $X$ seems to lie in understanding the required correction
  factor. The recent work of Morin \cite{Morin-2021-THH} may be particularly
  helpful here.

\item Lichtenbaum in his recent preprint \cite{Lichtenbaum-2021} gives a
  different special value conjecture, which up to a power of $2$ shold be
  equivalent to that of Flach and Morin \cite{Flach-Morin-2018}. It would be
  useful to explicitly compare the two conjectures.

\item It is likely that my results can be generalized to Weil-étale cohomology
  for $\ZZ$-constructible sheaves $\mathcal{F}$, in the spirit of
  \cite{Geisser-Suzuki-2020} and \cite{Adrien-Morin-2021}. I plan to consider
  these generalizations in the near future.

\item The construction of Weil-étale cohomolgy by Flach and Morin
  \cite{Flach-Morin-2018} uses the so-called \textbf{Artin--Verdier topology},
  a modification of the étale topology $X_\text{\'et}$, which is supposed to take
  care of the real places $X (\RR)$. A suitable construction is given in
  [ibid., \S 6], but \emph{only for proper and regular $X$}. The methods I use
  in \cite{Weil-etale-cohomology} circumvent this restriction, but at the cost
  of introducing an auxiliary cohomology theory $H^i_{fg} (X, \ZZ(n))$ which is
  not bounded, but can instead have finite $2$-torsion in arbitrarily high
  degrees $i \gg 0$.

  It would be interesting to work out the right construction of Artin--Verdier
  topology for any $X \to \Spec \ZZ$ separated and of finite type, and rework
  \cite{Weil-etale-cohomology} accordingly.

\item The actual construction of Weil-étale cohomology is not formulated in
  terms of separate groups $H^i_{W,c} (X, \ZZ (n))$, but in terms of complexes
  $R\Gamma_{W,c} (X, \ZZ (n))$. At present, these are defined (both in
  \cite{Flach-Morin-2018} and in \cite{Weil-etale-cohomology}) up to a
  \emph{non-unique} isomorphism in the derived category $\mathbf{D} (\ZZ)$, as a
  mapping fiber of certain canonical morphism in $\mathbf{D} (\ZZ)$. This is not
  much of a problem for the special value conjecture, since the determinants
  $\det_\ZZ R\Gamma_{W,c} (X, \ZZ (n)) = \det_\ZZ H^\bullet_{W,c} (X, \ZZ(n))$
  are uniquely defined, but it would be useful to find a more canonical
  definition for $R\Gamma_{W,c} (X, \ZZ (n))$.

  This could probably be remedied using dg-categories \cite{Toen-2011} or
  stable $\infty$-categories \cite{Lurie-stable-oo}.

\item In the above exposition, I have swept some details about regulators under
  the rug. I use the construction of Kerr, Lewis, and Müller-Stach from
  \cite{KLMS}. In general, the regulator has to do with complex points
  $X (\CC)$, and the regulator in \cite{KLMS} is defined for smooth and
  quasi-projective $X_\CC$. This is rather unfortunate, because my construction
  of Weil-étale cohomology $H^i_{W,c} (X, \ZZ (n))$ works for any $X$ that is
  separated and of finite type over $\Spec \ZZ$ (assuming finite generation of
  the corresponding motivic cohomology). It would be interesting to find
  suitable generalizations of the regulator for singular $X_\CC$, and connect
  them to my machinery.

  There are variants of motivic cohomology for singular complex varieties,
  defined in terms of hyperresolutions, e.g. Hanamura's ``Chow cohomology
  groups'' \cite{Hanamura-2000}, which allow us to formally extend the formula
  of Kerr--Lewis--Müller-Stach to the singular case. However, it seems that
  singular regulators have not been thoroughly considered for special value
  conjectures.

\item The Tamagawa number conjecture has a generalization, known as the
  \textbf{equivariant Tamagawa number conjecture} (\textbf{ETNC});
  see e.g. Flach's survey \cite{Flach-ETNC}. Similarly, Weil-étale cohomology
  should also have an ``equivariant refinement'', and it would be interesting to
  write this down for my construction and prove the compatibility with ETNC.
\end{enumerate}

These are a few specific questions that arise naturally from my previous
work. In general, Weil-étale cohomology is an active research topic and there
are still many open problems in this context.

\bibliographystyle{amsalpha-cust}
\bibliography{zeta}

\end{document}
