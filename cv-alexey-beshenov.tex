\documentclass{article}

\usepackage{fullpage}

\usepackage{graphicx}
\usepackage{wrapfig}

\usepackage[utf8]{inputenc}
\usepackage[spanish]{babel}

\usepackage{fontspec}
\setmainfont{Open Sans}
\setsansfont{IBM Plex Sans}
\setmonofont{IBM Plex Mono}

\usepackage[hidelinks]{hyperref}

\hypersetup{
    colorlinks,
    urlcolor={blue!80!black}
}

\usepackage{fontawesome}
\usepackage[misc]{ifsym}

\setlength{\parindent}{0pt}

\usepackage{sectsty}
\sectionfont{\fontsize{12}{15}\sffamily\bfseries}

\usepackage[dvipsnames]{xcolor}

\begin{document}

{\flushright\noindent\emph{Actualizado 10/11/2020}

}

\begin{center}
{\LARGE\sffamily\bf ALEXEY BESHENOV}

\vspace{0.5em}

\faPhoneSquare{} (+52)~473\,139\,4002 \quad
\faGlobe{} \href{https://cadadr.org/}{cadadr.org} \quad
\faEnvelope{} \href{mailto:alexey.beshenov@cimat.mx}{alexey.beshenov@cimat.mx} \\

\vspace{0.5em}

C. Privada San Martín \#10, Col. La Venada, 36030, Guanajuato, Gto.

\vspace{1em}

\rule{14cm}{1pt}
\end{center}

\vspace{1em}

\noindent \textbf{Fecha de nacimiento}: 24/02/1989, URSS \\
\textbf{Ciudadanía}: ruso

{\color{RoyalBlue}\section*{INTERESES ACADÉMICOS}}

Geometría algebraica, teoría de números.

\vspace{1em}

{\color{RoyalBlue}\section*{EDUCACIÓN}}

\begin{itemize}
\item \textbf{2014--2018}: \textbf{Universidad de Burdeos} (Francia),
  \textbf{Universidad de Leiden} (Países Bajos).

  Programa ALGANT DOC, beca de la Unión Europea.

  \textbf{PhD en matemáticas}.

  Tesis <<Zeta-values of arithmetic schemes at negative integers and Weil-étale cohomology>>,\\
  dirigida por Baptiste Morin (Burdeos) y Bas Edixhoven (Leiden).

  Defensa oficial: Leiden, 10/12/2018.

  Jurado:
  S.~Lichtenbaum (Brown University),
  N.~Ramachandran (University of Maryland),
  P.~Stevenhagen (Universiteit Leiden),
  Ph.~Cassou-Noguès (Université de Bordeaux),
  P.~Cassou-Noguès (Université de Bordeaux),
  R.~de Jeu (Vrije Universiteit Amsterdam),
  W.~van der Kallen (Universiteit Utrecht),
  H.~Lenstra (Universiteit Leiden).

\item \textbf{2012--2014}: \textbf{Universidad de Milán} (Italia),
  \textbf{Universidad de Burdeos} (Francia).

  Programa ALGANT Master, beca de la Unión Europea.

  \textbf{Maestro en Ciencias con orientación en matemáticas}.

  Tesis de maestría sobre la teoría K algebraica,
  dirigida por Boas Erez (Burdeos).

\item \textbf{2010--2012}: Universidad de la Academia de Ciencias de Rusia,
  San Petersburgo, Facultad de matemáticas e informática.

  \textbf{Maestro en Ciencias con orientación en informática teórica},
  diploma cum laude.

  Director de tesis: Dmitrii Pasechnik.

\item \textbf{2006--2010}: Universidad Estatal de Lipetsk, Rusia.

  \textbf{Licenciado en ciencias de computación y programación},
  diploma cum laude.
\end{itemize}

\pagebreak

{\color{RoyalBlue}\section*{POSICIONES}}

\begin{itemize}
\item \textbf{Octubre de 2019 -- noviembre de 2020}:
  \textbf{Centro de Investigación en Matemáticas (CIMAT)},
  Guanajuato, \textbf{investigador invitado}.

  Anfitriones: Xavier Gómez--Mont, Pedro Luis del Ángel.

\item \textbf{Febrero de 2018 -- agosto de 2019}:
  Universidad de El~Salvador,
  Facultad de Ciencias Naturales, Escuela de Matemáticas,
  \textbf{profesor invitado}.

  Colaboración con el Ministerio de Educación de El~Salvador,
  estancia con el fin de apoyar el programa de maestría en matemáticas.

  Clases para estudiantes de maestría y licenciatura, redacción de
  \href{https://cadadr.org/san-salvador/}{materiales didácticos}.
\end{itemize}

{\color{RoyalBlue}\section*{CONFERENCIAS}}

\begin{itemize}
\item \textbf{Febrero de 2020}: Coloquio Oaxaqueno de Matemáticas,
  IMUNAM, Unidad de Oaxaca.

\item \textbf{Diciembre de 2019}: First IMSA Conference,
  IMUNAM/CINVESTAV, CDMX.

\item \textbf{Noviembre de 2019}: Universidad Autónoma de Zacatecas
  (tres conferencias).

\item \textbf{Octubre de 2019}: XIII Taller de Álgebra y Topología,
  IMUNAM, Unidad de Cuernavaca.

\item \textbf{Octubre de 2019}: Seminario de geometría algebraica,
  CIMAT, Guanajuato.

\item \textbf{Mayo de 2019}: Conferencias Samuel Gitler,
  CINVESTAV, CDMX.

\item \textbf{Diciembre de 2017}:
  Algebra, geometry and number theory seminar, Leiden.
\end{itemize}

{\color{RoyalBlue}\section*{PUBLICACIONES}}

\begin{itemize}
\item A.~Beshenov, M.~Bilu, Yu.~Bilu, P.~Rath,
  \emph{\href{https://arxiv.org/abs/1408.1441}{Rational points on analytic varieties}},
  EMS Surv. Math. Sci. 2 (2015), no. 1, 109–130.

\item \emph{Weil-étale cohomology for n<0}, 2020, prepublicación,
  en preparación.
\end{itemize}

\pagebreak

{\color{RoyalBlue}\section*{EXPERIENCIA DOCENTE}}

\noindent\textbf{Cursos semestrales}

\vspace{0.5em}

\begin{itemize}
\item \textbf{Otoño de 2020}:
  \href{https://cadadr.org/cimat-tna/}{\textbf{Teoría de números algebraicos}},
  maestría en matemáticas básicas, CIMAT, Guanajuato.

\item \textbf{Primavera de 2019}:
  \href{https://cadadr.org/san-salvador/2019-groebner/}{\textbf{Álgebra conmutativa computacional (bases de Gröbner)}}
  para la maestría, Universidad de El~Salvador.

\item \textbf{Primavera de 2019}:
  \href{https://cadadr.org/san-salvador/2019-algebra/}{\textbf{Álgebra I (anillos y grupos)}}
  para la licenciatura, Universidad de El~Salvador.

\item \textbf{Primavera de 2018}:
  \href{https://cadadr.org/san-salvador/2018-08-algebra-conmutativa/}{\textbf{Álgebra conmutativa}}
  para la maestría, Universidad de El~Salvador.

\item \textbf{Otoño de 2018}:
  \href{https://cadadr.org/san-salvador/2018-algebra/}{\textbf{Álgebra II (anillos y campos)}}
  para la licenciatura, Universidad de El~Salvador.

\item \textbf{Primavera de 2018}:
  \href{https://cadadr.org/san-salvador/2018-algebra/}{\textbf{Álgebra I (grupos)}}
  para la licenciatura, Universidad de El~Salvador.
\end{itemize}

\noindent\textbf{Minicursos}

\begin{itemize}
\item \textbf{Otoño de 2019}:
  \href{https://cadadr.org/cimat-zeta/}{\textbf{En torno de las funciones zeta aritméticas}},
  CIMAT, Guanajuato.

\item \textbf{Agosto de 2019}:
  \href{https://cadadr.org/san-salvador/2019-esquemas/}{\textbf{Teoría de esquemas}}
  para la maestría, Universidad de El~Salvador.

\item \textbf{Noviembre de 2018}:
  \href{https://cadadr.org/san-salvador/2018-cp-tne/reciprocidad-cuadratica.pdf}{\textbf{La ley de reciprocidad cuadrática}}
  para la licenciatura, Universidad de El~Salvador.

\item \textbf{Julio de 2018}:
  \href{https://cadadr.org/san-salvador/2018-07-reciprocidad/reciprocidad.pdf}{\textbf{Las leyes de reciprocidad de Gauss a Artin}},
  Universidad de El~Salvador.

\item \textbf{Junio de 2018}:
  \href{https://cadadr.org/san-salvador/2018-06-categorias/}{\textbf{Teoría de categorías}}
  para la maestría, Universidad de El~Salvador.

\item \textbf{Abril de 2018}:
  \href{https://cadadr.org/san-salvador/2018-04-numeros-p-adicos/}{\textbf{Números p-ádicos}}
  para la maestría, Universidad de El~Salvador.

\item \textbf{Febrero de 2018}:
  \href{https://cadadr.org/san-salvador/2017-02-bernoulli/}{\textbf{Números de Bernoulli}},
  Universidad de El~Salvador.

\item \textbf{Agosto–septiembre de 2016}:
  \href{https://cadadr.org/san-salvador/2016-08-homo/}{\textbf{Álgebra homológica}},
  Universidad de El~Salvador.
\end{itemize}

{\color{RoyalBlue}\section*{IDIOMAS}}

Ruso (nativo); español, inglés (fluido); francés, italiano (intermedio).

\vspace{1em}

{\color{RoyalBlue}\section*{REFERENCIAS}}

\begin{itemize}
\item Baptiste Morin (Université de Bordeaux)
\item Xavier Gómez--Mont (CIMAT)
\item Pedro Luis del Ángel (CIMAT)
\end{itemize}

\end{document}
